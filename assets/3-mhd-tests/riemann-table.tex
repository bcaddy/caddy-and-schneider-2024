% Can add a * after "deluxetable" in begin and end to let it span 2 columns

%% The values (usually only l,r and c) in the last part of
%% \begin{deluxetable}{} command tell LaTeX how many columns
%% there are and how to align them.
\begin{deluxetable*}{lccccccccccccccccc}
    \label{table:riemann}
    
    %% Over-ride the default font size
    %% Use Default (12pt)
    
    %% Use \tablewidth{?pt} to over-ride the default table width.
    %% If you are unhappy with the default look at the end of the
    %% *.log file to see what the default was set at before adjusting
    %% this value.
    
    %% This is the title of the table.
    \tablecaption{Riemann Problem Initial Conditions}
    
    %% The \tablehead gives provides the column headers.  It
    %% is currently set up so that the column labels are on the
    %% top line and the units surrounded by ()s are in the 
    %% bottom line.  You may add more header information by writing
    %% another line between these lines. For each column that requries
    %% extra information be sure to include a \colhead{text} command
    %% and remember to end any extra lines with \\ and include the 
    %% correct number of &s.
    \tablehead{\colhead{Riemann Problem} & \colhead{$\gamma$} & \colhead{$t_{max}$} & \colhead{$B_x$} & 
    \colhead{$\rho_L$} & \colhead{$P_L$} & \colhead{$v_{x,L}$} & \colhead{$v_{y,L}$} & \colhead{$v_{z,L}$} & \colhead{$B_{y,L}$} & \colhead{$B_{z,L}$} & 
    \colhead{$\rho_R$} & \colhead{$P_R$} & \colhead{$v_{x,R}$} & \colhead{$v_{y,R}$} & \colhead{$v_{z,R}$} & \colhead{$B_{y,R}$} & \colhead{$B_{z,R}$}}
    
    %% All data must appear between the \startdata and \enddata commands
    \startdata
    % Riemann Problem      gamma           t_max  B_x                       rho_L  P_L    V_x,L  V_y,L  V_z,L B_y,L                       B_z,L                     rho_R   P_R    V_x,R   V_y,R V_z,R B_y,R                     B_z,R
    Brio \& Wu           & 2             & 0.1  & 0.75                    & 1    & 1    & 0    & 0    & 0   & 1                         & 0                       & 0.128 & 0.1  & 0     & 0   & 0   & -1                      & 0                       \\
    Dai \& Woodward      & $\frac{5}{3}$ & 0.2  & $\frac{2}{\sqrt{4\pi}}$ & 1.08 & 0.95 & 1.2  & 0.01 & 0.5 & $\frac{3.6}{\sqrt{4\pi}}$ & $\frac{2}{\sqrt{4\pi}}$ & 1     & 1    & 0     & 0   & 0   & $\frac{4}{\sqrt{4\pi}}$ & $\frac{2}{\sqrt{4\pi}}$ \\
    Ryu \& Jones 1a      & $\frac{5}{3}$ & 0.08 & $\frac{5}{\sqrt{4\pi}}$ & 1    & 20   & 10   & 0    & 0   & $\frac{5}{\sqrt{4\pi}}$   & 0                       & 1     & 1    & -10   & 0   & 0   & $\frac{5}{\sqrt{4\pi}}$ & 0                       \\
    Ryu \& Jones 4d      & $\frac{5}{3}$ & 0.16 & 0.7                     & 1    & 1    & 0    & 0    & 0   & 0                         & 0                       & 0.3   & 0.2  & 0     & 0   & 1   & 1                       & 0                       \\
    Einfeldt Rarefaction & 1.4           & 0.16 & 0                       & 1    & 0.45 & -2   & 0    & 0   & 0.5                       & 0                       & 1     & 0.45 & 2     & 0   & 0   & 0.5                     & 0                       \\
    \enddata
    
    %% Include any \tablenotetext{key}{text}, \tablerefs{ref list},
    %% or \tablecomments{text} between the \enddata and 
    %% \end{deluxetable} commands
    
    %% General table comment marker
    \tablecomments{The $L/R$ subscripts indicate that it is the left/right state. $B_x$ is always the same in both states.}
    
\end{deluxetable*}