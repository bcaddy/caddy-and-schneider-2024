% - [x] disclaimer, this is not new and based on all these sources. We order it slightly differently to work on GPUs so we will outline it here
% - [x] provide summary of integrator first then provide details. Then there's space in the first section to discuss details of the implementation
%  - [ ] How was this implemented on GPUs and what made it different
%    - [ ] what does Cholla do when I press go?
%      - [ ] allocations, data movement, etc
%  - [ ] here's the details on all the math

\section{Methods}
\label{sec:methods}

\subsection{Overview of the VL+CT Integrator}
\label{sec:vlct-summary}

The MHD integrator in Cholla uses the VL+CT (Van Leer plus Constrained Transport) integrator introduced in \cite{stone_2009} and used in Athena and Athena++ since then. The specific algorithm is from \cite{stone_2009} with some clarifying details from \cite{gardiner_2005,gardiner_unsplit_2008,stone_athena_2008} and the piecewise parabolic method from \cite{felker_2020}. We also use the HLLD Riemann solver from \cite{hlld_2005}.

Overall, the integrator is very similar in structure to the Van Leer integrator, though with significant new additions for Constrained Transport (CT). Constrained transport treats the magnetic field as surface averaged quantity centered at cell interfaces rather than a volume averaged, cell centered quantity like the hydro variables. Also, each face stores only the magnetic field perpendicular to that face, i.e. the $x,i+1/2,j,k$ face store the $B_{x,i+1/2,j,k}$ magnetic field. The magnetic field is then updated with edge averaged EMFs (Electromotive Force) computed from the magnetic flux returned by the Riemann solver. By updating with EMF you automatically fulfill the divergence free condition for magnetic fields so no magnetic monopoles will appear, assuming that the initial conditions are divergence free.

The steps of the VL+CT integrator are as follows; more detailed discussion of each step is in section \ref{vlct:header}.

\begin{enumerate}
    \item Compute the time step
    \item Perform piecewise constant reconstruction of the interface state
    \item Perform a Riemann solve on the reconstructed interface state
    \item Compute the edge centered electric fields
    \item Update the $t=n$ conserved variables to $t=n+\delta t/2$ using the computed fluxes and electric fields
    \item Perform piecewise linear or parabolic reconstruction of the interface state
    \item Perform a Riemann solve on the higher order reconstructed interface state
    \item Compute the edge centered electric fields
    \item Update the $t=n$ conserved variables to $t=n+\delta t$ using the half time step fluxes and electric fields
\end{enumerate}

% ===========================================================================
\subsection{VL+CT Integrator in Detail}
\label{vlct:header}

Throughout this section \cite{stone_2009} will be referred to as SG09.

\subsubsection{Step 1: Compute the Time Step}
\label{vlct:dt}

The first thing we need to do is compute the time step which is done with this equation (SG09 eqn. 31)

\begin{equation}
    \delta t = C_{CFL} \min \left(
        \frac{\delta x}{\mid v^n_{x,i,j,k} \mid + C^n_{f,i,j,k}},
        \frac{\delta y}{\mid v^n_{y,i,j,k} \mid + C^n_{f,i,j,k}},
        \frac{\delta z}{\mid v^n_{z,i,j,k} \mid + C^n_{f,i,j,k}}
    \right).
\end{equation}

Where $\delta t$ is the time step, $C_{CFL} \leq 0.5$ is the CFL number, $\mid v^n_{l,i,j,k}\mid $ is the magnitude of the velocity in the $l$ direction velocity in the ${i,j,k}$ cell, and $C^n_f $ is the fast magnetosonic wave-speed computed using \emph{cell centered} values. This computed the minimum crossing time of a wave in a specific cell and then a global reduction is performed to find the minimum in the entire grid; that minimum is used as the time step $\delta t$.

Computing the cell centered magnetic field is done with a direct average of the face centered values (SG09 eqns. 15-17). These cell centered values for the magnetic field will the used several times throughout the integrator. In CPU based codes it's typically more efficient to compute these cell centered magnetic fields once and save them since CPU based codes are usually compute limited. Cholla is generally limited instead by GPU memory and so these values are recomputed when needed rather than allocating another large chunk of memory.

\begin{equation}
    \begin{aligned}
        B^n_{x,i,j,k,} = \frac{1}{2} \left( B^n_{x,i+1/2,j,k} + B^n_{x,i-1/2,j,k} \right) \\
        B^n_{y,i,j,k,} = \frac{1}{2} \left( B^n_{y,i,j+1/2,k} + B^n_{y,i,j-1/2,k} \right) \\
        B^n_{z,i,j,k,} = \frac{1}{2} \left( B^n_{z,i,j,k+1/2} + B^n_{z,i,j,k-1/2} \right) \\
    \end{aligned}
\end{equation}

\subsubsection{Step 2: First Order Reconstruction}
\label{vlct:first-order-reconstruction}

Reconstructing the interface states at first order, or piecewise constant (PCM), is done by setting the primitive interface states values to the same value as the cell. PCM is far too diffusive for most simulations but is excellent for debugging or some rare problems where the extra diffusion is desirable.

\begin{equation}
    \vec{W}_{L, i+1/2} = \vec{W}_{R, i-1/2} = \vec{W}_{i}
\end{equation}

where $ \vec{W}_{L/R, i\pm1/2} $ is the state on the left or right side of the cell

\subsubsection{Step 3: First Riemann Solve}
\label{vlct:first-riemann-solve}

The next step is the solve the Riemann problem with the first order interface states. We use the HLLD Riemann Solver introduced in \cite{hlld_2005}. This is identical to the standard Van Leer integrator for the hydrodynamic variables; the magnetic field is more complex since the magnetic field is stored as surface averages. The longitudinal field (i.e. the field that is stored at the interface itself) does not require reconstruction and can be used directly. The transverse fields (i.e. the fields parallel to the interface) are reconstructed using an average as shown in section \ref{vlct:dt}.

The magnetic fluxes that are returned by the Riemann solver are the face centered EMFs (section 5.3 of \cite{stone_athena_2008}), however it's not immediately clear which face centered EMFs they are. Table \ref{table:emf} details which flux is which EMF.

\begin{table}[!ht]
    \centering
        \caption{Magnetic Flux to Face Centered EMF}
    \begin{tabular}{|l|l|l|l|}
    \hline
        HLLD solve Direction & Equation for Magnetic Flux & Eqn. as a Cross Product & EMF \\ \hline
        $ X $ & $ V_x B_y - B_x V_y $ & $  (V \times B)_z $ & $ -\varepsilon_z $ \\ \hline
        $ X $ & $ V_x B_z - B_x V_z $ & $ -(V \times B)_y $ & $  \varepsilon_y $ \\ \hline
        $ Y $ & $ V_x B_y - B_x V_y $ & $  (V \times B)_z $ & $ -\varepsilon_x $ \\ \hline
        $ Y $ & $ V_x B_z - B_x V_z $ & $ -(V \times B)_y $ & $  \varepsilon_z $ \\ \hline
        $ Z $ & $ V_x B_y - B_x V_y $ & $  (V \times B)_z $ & $ -\varepsilon_y $ \\ \hline
        $ Z $ & $ V_x B_z - B_x V_z $ & $ -(V \times B)_y $ & $  \varepsilon_x $ \\ \hline
    \end{tabular}
    \tablecomments{The directions used here are relative to the internal workings of the HLLD solver. Since the HLLD solver is inherently 1D we run it three times, once for each direction. So in the case where the solver is running in the Y direction the solver's Y field is actually the Z field and the solvers Z field is actually the X field, cyclically extended for the Z direction}
    \label{table:emf}
\end{table}

\subsubsection{Step 4: Compute the Constrained Transport EMF}
\label{vlct:emf}

Now we need to calculate the CT EMF. These fields are \emph{line averaged} along each cell edge. These line averaged fields are constructed by a complex average of the face centered EMF and their slopes which are given by the flux of the magnetic field from the Riemann solve in the previous step.

Technically CT uses the magnetic flux as the conservative variable and EMF to update it. However, those values only differ from the magnetic flux density (i.e. the magnetic field) and the electric field respectively by factors of unit length we can treat them as the same thing, the same way we treat density and mass as the same for the hydro fields \cite{stone_athena_2008}. As such we use the magnetic field and electric field to evolve the grid. Electric fields have the proper units to evolve the magnetic flux density, $ B $, whereas EMF has the proper units to evolve the magnetic flux.

On any face there are two non-zero electric fields; both transverse to the face. The most important thing to note is that the component that is used to calculate the field along a given edge is the component that is parallel to that edge. i.e. if the edge points along the $ z $-direction then you use the field pointing along the $ z $-direction, not the field in the $ x $ or $ y $ direction.

We also need cell centered electric fields (reference fields) for computing the slopes. This reference field can be computed with the following cross product (\cite{gardiner_2005} eq. 74).

\begin{equation}
    \mathcal{E}_{i,j,k}^{ref,n} = - \left( \vec{v}^{n}_{i.j.k} \times \vec{B}^{n}_{i.j.k} \right)
\end{equation}

Example for computing the CT electric field in the $ z $-direction. The other direction can be computed by just substituting out the $ z $ index with $ x $ or $ y $ and changing the derivatives appropriately, see the figure \ref{fig:emf-graph} for more details. Equations \ref{eqn:emf-edge}, \ref{eqn:emf-upwind-slope}, and \ref{eqn:emf-slope} are from SG09, equations 22-24. In the original paper there are a few sign errors which have been corrected in the document.

\begin{equation}
    \label{eqn:emf-edge}
    \begin{aligned}
        \mathcal{E}_{z, i-1/2, j-1/2, k} = \frac{1}{4} \left(
              \mathcal{E}_{z, i-1/2, j, k}
            + \mathcal{E}_{z, i, j-1/2, k}
            + \mathcal{E}_{z, i-1/2, j-1, k}
            + \mathcal{E}_{z, i-1, j-1/2, k}\right) \\
        + \frac{\delta y}{8} \left( \left( \frac{\partial \mathcal{E}_z }{\partial y} \right)_{i-1/2, j-1/4, k} + \left(  \frac{\partial \mathcal{E}_z }{\partial y} \right)_{i-1/2, j-3/4, k} \right) \\
        + \frac{\delta x}{8} \left( \left( \frac{\partial \mathcal{E}_z }{\partial x} \right)_{i-1/4, j-1/2, k} + \left(  \frac{\partial \mathcal{E}_z }{\partial x} \right)_{i-3/4, j-1/2, k} \right)
    \end{aligned}
\end{equation}

Where the derivatives are computed using the upwinded slope as follows (SG09 eqn. 23).

\begin{equation}
    \label{eqn:emf-upwind-slope}
    \left( \frac{\partial \mathcal{E}_z }{\partial y} \right)_{i-1/2, j-1/4, k} =
        \begin{cases}
            \left( \frac{\partial \mathcal{E}_z }{\partial y} \right)_{i-1, j-1/4, k} & \text{for} \; v_{x, i-1/2} > 0
            \\
            \\
            \left( \frac{\partial \mathcal{E}_z }{\partial y} \right)_{i, j-1/4, k} & \text{for} \; v_{x, i-1/2} < 0
            \\
            \\
            \frac{1}{2} \left( \left( \frac{\partial \mathcal{E}_z }{\partial y} \right)_{i-1, j-1/4, k} + \left( \frac{\partial \mathcal{E}_z }{\partial y} \right)_{i, j-1/4, k} \right) & \text{otherwise}
            \\
            \\
        \end{cases}
\end{equation}

Where, for example, the derivatives are given by (SG09 eqn. 24)

\begin{equation}
    \label{eqn:emf-slope}
    \left( \frac{\partial \mathcal{E}_z }{\partial y} \right)_{i, j-1/4, k} =
    2 \left( \frac{\mathcal{E}_{z,i,j-1/2,k} - \mathcal{E}_{z,i,j,k}^{ref}}{\delta y} \right).
\end{equation}

The easiest way to see how to compute these derivatives is with the Figure \ref{fig:emf-graph} based off of Figure 5 in \cite{stone_athena_2008}. Each edge requires 4 derivatives and they are computed as differences between a reference state and an edge state.

\begin{figure}[ht!]
    \epsscale{0.5}
    \plotone{assets/2-methods/CT-edge-field-figures-X.pdf}
    \plotone{assets/2-methods/CT-edge-field-figures-Y.pdf}
    \plotone{assets/2-methods/CT-edge-field-figures-Z.pdf}
    \caption{2D slices in all three planes showing the location of the fluxes, edge EMFs, and derivatives. Based on Figure 5 of \cite{stone_athena_2008}.}
    \label{fig:emf-graph}
\end{figure}

\subsubsection{Step 5. Perform the Half Time-step Update}
\label{vlct:half-dt-update}

Update all the hydro variables using this equation (SG09 eqn. 6)

\begin{equation}
    \begin{aligned}
        \vec{U}^{n+1/2}_{i,j,k} = \vec{U}^{n}_{i,j,k}
        - \frac{\delta t}{\delta x} \left( \vec{F}^n_{x,i+1/2,j,k} - \vec{F}^n_{x,i-1/2,j,k} \right) \\
        - \frac{\delta t}{\delta y} \left( \vec{F}^n_{y,i+1/2,j,k} - \vec{F}^n_{y,i-1/2,j,k} \right) \\
        - \frac{\delta t}{\delta z} \left( \vec{F}^n_{z,i+1/2,j,k} - \vec{F}^n_{z,i-1/2,j,k} \right).
    \end{aligned}
\end{equation}

Update the magnetic field using these equations (SG09 eqns. 10-12)

\begin{equation}
    \begin{aligned}
        B^{n+1/2}_{x,i-1/2,j,k} = B^{n}_{x,i-1/2,j,k}
        + \frac{\delta t}{\delta z} \left( \mathcal{E}^n_{y,i-1/2,j,k+1/2} - \mathcal{E}^n_{y,i-1/2,j,k-1/2} \right) \\
        - \frac{\delta t}{\delta y} \left( \mathcal{E}^n_{z,i-1/2,j+1/2,k} - \mathcal{E}^n_{z,i-1/2,j-1/2,k} \right)
    \end{aligned}
\end{equation}

\begin{equation}
    \begin{aligned}
        B^{n+1/2}_{y,i,j-1/2,k} = B^{n}_{y,i,j-1/2,k}
        + \frac{\delta t}{\delta x} \left( \mathcal{E}^n_{z,i+1/2,j-1/2,k} - \mathcal{E}^n_{z,i-1/2,j-1/2,k} \right) \\
        - \frac{\delta t}{\delta z} \left( \mathcal{E}^n_{x,i,j-1/2,k+1/2} - \mathcal{E}^n_{x,i,j-1/2,k-1/2} \right)
    \end{aligned}
\end{equation}

\begin{equation}
    \begin{aligned}
        B^{n+1/2}_{z,i,j,k-1/2} = B^{n}_{z,i-1/2,j,k}
        + \frac{\delta t}{\delta y} \left( \mathcal{E}^n_{x,i,j+1/2,k-1/2} - \mathcal{E}^n_{x,i,j-1/2,k-1/2} \right) \\
        - \frac{\delta t}{\delta x} \left( \mathcal{E}^n_{y,i+1/2,j,k-1/2} - \mathcal{E}^n_{y,i-1/2,j,k-1/2} \right).
    \end{aligned}
\end{equation}

\subsubsection{Step 6. Half Time-step Second Order Reconstruction}
\label{vlct:higher-order-reconstruction}

Now we need to perform the high order reconstruction. I will show the algorithm for a second order, Piecewise Linear Method (PLM), reconstruction. Cholla currently implements piecewise constant, piecewise linear and piecewise parabolic reconstruction with limiting in the characteristic variables for MHD. The piecewise parabolic method that Cholla utilizes is discussed in detail in \cite{felker_2020}. Note that at a given face only the transverse components of the electric field need to be reconstructed. The longitudinal component is already given at the face.

\begin{enumerate}
    \item Compute the primitive variables
    
    \item Compute the left, right, centered, and Van Leer differences in the primitive variables
    \begin{align} 
        \delta \vec{w}_{L,i} = \vec{w_{i}} - \vec{w_{i-1}} \\ 
        \delta \vec{w}_{R,i} = \vec{w_{i+1}} - \vec{w_{i}} \\
        \delta \vec{w}_{C,i} = \frac{\vec{w_{i+1}} - \vec{w_{i-1}}}{2} \\
        \delta \vec{w}_{VL,i} = 
        \begin{cases}
            \frac{2 \vec{w}_{L,i} \vec{w}_{R,i}}{\vec{w}_{L,i} +\vec{w}_{R,i}} ,& \text{if } \vec{w}_{L,i} \vec{w}_{R,i} > 0\\
            0,              & \text{otherwise}
        \end{cases}
    \end{align}
    
    \item Project the slopes into the characteristic variables $\vec{a}$ using the eigenvectors detailed in \cite{stone_athena_2008}. Note that to maintain mathematical consistency we use the eigenvectors of the $\vec{w_{i}}$ cell for all three slopes and the later projection back to primitive variables.
    \item Apply monotonicity constraints to the characteristic differences to ensure that the reconstruction is total variation diminishing (TVD). We used this form from \cite{leveque2002finite}.
    \begin{equation}
        \delta \vec{a}_{m,i} = 
        \begin{cases}
             \text{sgn}(\vec{a}_{C,i})\min(2\abs{\vec{a}_{L,i}},2\abs{\vec{a}_{R,i}},\abs{\vec{a}_{C,i}},\abs{\vec{a}_{VL,i}}),& \text{if } \vec{a}_{L,i} \vec{a}_{R,i} > 0\\
            0,              & \text{otherwise}
        \end{cases}
    \end{equation}
    \item Project the characteristic slopes back into the primitive variables $\vec{a}$ using the eigenvectors detailed in \cite{stone_athena_2008}.
    \item Then compute the interface states $\vec{W}_{L, i+1/2}$ and $\vec{W}_{R, i-1/2}$ (SG09 eqns. 20 \& 21)
\end{enumerate}

    \begin{equation}
        \begin{aligned}
            \vec{W}_{L, i+1/2} = \vec{W}_{i} + \frac{\delta \vec{W}_{i}^m}{2} \\
            \vec{W}_{R, i-1/2} = \vec{W}_{i} - \frac{\delta \vec{W}_{i}^m}{2} \\
        \end{aligned}
    \end{equation}

    where $ \vec{W}_{L/R, i\pm1/2} $ is the state on the left or right side
    of the cell and $ \delta \vec{W}_{i}^m $ is the monotonically limited
    primitive slope.

\subsubsection{Step 7. Second Riemann Solve}
\label{vlct:2nd-riemann-solve}

Solve the Riemann problem again with the half time step MHD interface states from step 6.

\subsubsection{Step 8. Compute the Constrained Transport Electric Fields}
\label{vlct:2nd-emf}

Repeat step 3 but using the fluxes from the second Riemann solve and the half time step MHD variables.

\subsubsection{Step 9. Perform the Full Time-step Update}
\label{vlct:full-dt-update}

Update all the hydro variables using the below equations. Note that this is almost identical to step 3 except we're updating the $ t = n $ state with the $ t=n+1/2 $ fluxes/CT fields instead of the $ t=n $ fluxes/CT fields (SG09 eqn. 6).

\begin{equation}
    \begin{aligned}
        \vec{U}^{n+1}_{i,j,k} = \vec{U}^{n}_{i,j,k}
        - \frac{\delta t}{\delta x} \left( \vec{F}^{n+1/2}_{x,i+1/2,j,k} - \vec{F}^{n+1/2}_{x,i-1/2,j,k} \right) \\
        - \frac{\delta t}{\delta y} \left( \vec{F}^{n+1/2}_{y,i+1/2,j,k} - \vec{F}^{n+1/2}_{y,i-1/2,j,k} \right) \\
        - \frac{\delta t}{\delta z} \left( \vec{F}^{n+1/2}_{z,i+1/2,j,k} - \vec{F}^{n+1/2}_{z,i-1/2,j,k} \right).
    \end{aligned}
\end{equation}

Update the magnetic field using these equations (SG09 eqns. 10-12)

\begin{equation}
    \begin{aligned}
        B^{n+1}_{x,i-1/2,j,k} = B^{n}_{x,i-1/2,j,k}
        + \frac{\delta t}{\delta z} \left( \mathcal{E}^{n+1/2}_{y,i-1/2,j,k+1/2} - \mathcal{E}^{n+1/2}_{y,i-1/2,j,k-1/2} \right) \\
        - \frac{\delta t}{\delta y} \left( \mathcal{E}^{n+1/2}_{z,i-1/2,j+1/2,k} - \mathcal{E}^{n+1/2}_{z,i-1/2,j-1/2,k} \right)
    \end{aligned}
\end{equation}

\begin{equation}
    \begin{aligned}
        B^{n+1}_{y,i,j-1/2,k} = B^{n}_{y,i,j-1/2,k}
        + \frac{\delta t}{\delta x} \left( \mathcal{E}^{n+1/2}_{z,i+1/2,j-1/2,k} - \mathcal{E}^{n+1/2}_{z,i-1/2,j-1/2,k} \right) \\
        - \frac{\delta t}{\delta z} \left( \mathcal{E}^{n+1/2}_{x,i,j-1/2,k+1/2} - \mathcal{E}^{n+1/2}_{x,i,j-1/2,k-1/2} \right)
    \end{aligned}
\end{equation}

\begin{equation}
    \begin{aligned}
        B^{n+1}_{z,i-1/2,j,k} = B^{n}_{z,i-1/2,j,k}
        + \frac{\delta t}{\delta y} \left( \mathcal{E}^{n+1/2}_{x,i,j+1/2,k-1/2} - \mathcal{E}^{n+1/2}_{x,i,j-1/2,k-1/2} \right) \\
        - \frac{\delta t}{\delta x} \left( \mathcal{E}^{n+1/2}_{y,i+1/2,j,k-1/2} - \mathcal{E}^{n+1/2}_{y,i-1/2,j,k-1/2} \right).
    \end{aligned}
\end{equation}

\subsubsection{Step 10. Increment the Time by \texorpdfstring{$\delta t$}{dt}}
\label{vlct:increment-time}

Increment the time by $\delta t$, perform boundary conditions, other physics, etc. then restart at step 1.
