\section{Summary}
\label{sec:summary}

We have presented the MHD extension to Cholla, a massively parallel, GPU native, astrophysical simulation code. MHD in Cholla uses the Van Leer plus Constrained Transport MHD integrator \citep{stone_2009}, the HLLD Riemann solver [citation], and includes multiple high order reconstruction methods to model numerical solutions to the Eulerian ideal MHD equations on a static mesh.

Designing the MHD extension to run on GPUs required some significant changes to the implementation compared to a CPU code and elicited challenges working within the limits of GPUs. The entire grid must stay in GPU memory to minimize the amount of data being passed between the CPU and GPU. GPUs also require specific memory layouts and a greater focus on reducing memory usage, even at the cost of increased computation. As demonstrated in Section \ref{sec:mhd-perf-tests}, these optimizations, and the highly parallel nature of GPUs, makes MHD in Cholla extremely fast, over 160 million cell updates per GPU-second even when scaled up to 74,088 GPUs for a total of 12.4 trillion cell updates per second (see Figure \ref{fig:scaling-weak-efficiency}).

We also present a suite of canonical MHD tests in Section \ref{sec:mhd-tests}. These tests demonstrate the accuracy of Cholla for a broad range of problems, and show that the MHD algorithm does an excellent job of maintaining the divergence free condition even in highly challenging settings. 

To accommodate the increasing complexity of Cholla and facilitate multiple simultaneous development efforts,  we have also added a structured testing framework, described in Section \ref{sec:testing-framework}. This framework is based on  GoogleTest\footnote{https://github.com/google/googletest}, along  with some custom testing tools, which enable running tests that range from single function unit tests to massively parallel system tests across the scale of an entire cluster. We have further integrated these tests with GitHub Actions\footnote{https://github.com/features/actions} to run formatting, static analysis, and builds with various configuration along with Jenkins\footnote{https://www.jenkins.io} running on local resources.

% Future work
In the future we expect to apply this code to a high resolution simulation of a Milky Way like galaxy with $\approx 10,000^3$ cells and resolution on the order of a single parsec to study the impact of magnetic fields on the galactic winds and the dynamics of the galactic dynamo.
