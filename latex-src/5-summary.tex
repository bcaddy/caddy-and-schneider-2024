\section{Summary}
\label{sec:summary}

We have presented the MHD extension to Cholla, a massively parallel, GPU native, astrophysical simulation code. MHD in Cholla uses the Van Leer plus Constrained Transport MHD integrator (VL+CT) \citep{stone_2009}, the HLLD Riemann solver \citep{hlld_2005}, and includes multiple high order reconstruction methods to model numerical solutions to the Eulerian ideal MHD equations on a static mesh.

We showed the modifications required to implement MHD on GPUs compared to CPUs and discussed challenges working within the limits of GPUs. One major challenge was moving computational work and data storage from the CPU to the GPU. Previous versions of Cholla did some of the computation on the CPU and used CPU memory to effectively expand GPU memory. This required regularly copying data between the CPU and the GPU which became a performance bottleneck. The current version of Cholla maintains all the data, and the vast majority of the work, on the GPU which has led to a considerable speedup. As demonstrated in Section \ref{sec:mhd-perf-tests}, these optimizations combined with the highly parallel nature of GPUs make MHD in Cholla extremely fast, with over 200 million cell updates per GPU-second when running on a single GPU. Cholla also demonstrates excellent weak scaling, and maintains a performance of over 160 million cell updates per GPU-second when scaled up to 74,088 GPUs; a total of 12.4 trillion cell updates per second (see Figure \ref{fig:scaling-weak-efficiency}).

We have also presented a suite of canonical MHD tests in Section \ref{sec:mhd-tests}. These tests demonstrate the accuracy of Cholla across a broad range of problems. They also demonstrate that the VL+CT MHD algorithm does an excellent job of maintaining the divergence free condition even in highly challenging settings. 

To accommodate the increasing complexity of Cholla and facilitate multiple simultaneous development efforts, we have also added a structured testing framework, described in Section \ref{sec:testing}. This framework is based on GoogleTest, augmented with custom testing tools. This approach enables running tests that range from single function unit tests to massively parallel system tests across the scale of an entire cluster within the same framework. We have further integrated these tests with GitHub Actions to run formatting, static analysis, and builds with various physics configurations along with Jenkins running on local resources.

% Future work
Cholla is free and open source software available at \url{https://github.com/cholla-hydro/cholla}. As shown in this work, it is designed to be modular to further accommodate future additional physics, and we welcome new development efforts. We hope that this work will be a resource for the broader astrophysics simulation community.
