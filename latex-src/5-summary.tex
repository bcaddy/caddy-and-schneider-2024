\section{Summary}
\label{sec:summary}

We have presented the MHD extension to Cholla, a massively parallel, GPU native, astrophysical simulation code. MHD in Cholla uses the Van Leer plus Constrained Transport MHD integrator \citep{stone_2009}, the HLLD Riemann solver [citation], and includes multiple high order reconstruction methods to model numerical solutions to the Eulerian ideal MHD equations on a static mesh.

We showed the modifications required to implement MHD on GPUs compared to a CPUs and discussed challenges working within the limits of GPUs. Making the grid entirely GPU resident enabled a considerable speedup compared to moving the grid between GPU and CPU memory. As demonstrated in Section \ref{sec:mhd-perf-tests}, these optimizations, and the highly parallel nature of GPUs, makes MHD in Cholla extremely fast, over 160 million cell updates per GPU-second even when scaled up to 74,088 GPUs for a total of 12.4 trillion cell updates per second (see Figure \ref{fig:scaling-weak-efficiency}).

We also present a suite of canonical MHD tests in Section \ref{sec:mhd-tests}. These tests demonstrate the accuracy of Cholla for a broad range of problems, and show that the MHD algorithm does an excellent job of maintaining the divergence free condition even in highly challenging settings. 

To accommodate the increasing complexity of Cholla and facilitate multiple simultaneous development efforts,  we have also added a structured testing framework, described in Section \ref{sec:testing}. This framework is based on  GoogleTest, along with some custom testing tools, which enable running tests that range from single function unit tests to massively parallel system tests across the scale of an entire cluster. We have further integrated these tests with GitHub Actions to run formatting, static analysis, and builds with various configuration along with Jenkins running on local resources.

% Future work
Cholla is free and open source software available at \url{https://github.com/cholla-hydro/cholla}. It is very extensible as this work has shown and we hope that this work will be a resource to the astrophysics community.
