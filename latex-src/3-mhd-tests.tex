\section{MHD Tests}
\label{sec:mhd-tests}

\subsection{Linear Wave Convergence}
\label{sec:lwc}

The propagation of the four MHD linear waves provide an excellent quantitative measure of the accuracy of computation methods \citep{stone_2009}. The tests use a domain of $1.0\times1.0\times1.0$ and a resolution of $N\times16\times16$ where $N$ goes from 16 to 512 in powers of 2. The equation of the wave is

\begin{equation}
    q = \overline{q} + A R_w \sin{\frac{2\pi x}{\lambda}}
\end{equation}

where $q$ is the conserved variable, $\overline{q}$ is the mean background state, $A=10^{-6}$ is the amplitude of the wave, $R_w$ is the right eigenvector in conserved variables for the wave mode $w$, $x$ is the position, and $\lambda=1$ is the wavelength of the wave. The adiabatic index $\gamma$ is $5/3$ and the background state is: 
$\overline{\rho}=1.0$,
$\overline{v_x}=\overline{v_y}=\overline{v_z}=0$ (except for the contact wave where $\overline{v_x} = 1$),
$\overline{P}=1/\gamma$,
$\overline{B_x}=1$,
$\overline{B_y}=1.5$,
$\overline{B_z}=0$ 
and the right eigenvectors for this state are given in Appendix A of \cite{gardiner_unsplit_2008}. 

The wave is propagated for one period and then the error between the initial and final state is the L2 norm of the L1 error vector. First we compute the L1 norm of the absolute difference for each conserved variable between the initial and final state

\begin{equation}
    \delta q_s = \frac{1}{n_x n_y n_z} \sum_{i,j,k} \mid q^f_{i,j,k,s} - q^i_{i,j,k,s} \mid
\end{equation}

where $q_s$ is a specific conserved variable. We then compute the L2 norm of this vector of L1 norms

\begin{equation}
    \mid \mid \delta q \mid \mid = \sqrt{\sum_s \left( \delta q_s \right)^2}.
\end{equation}

These L2 errors are plotted in \autoref{fig:linear-wave-convergence} for both the PLM and PPM reconstructions. The results are comparable to the results in \cite{stone_2009} and demonstrate the expected second order convergence for this second order method. Using PPM does improve the accuracy of the method but maintains the second order convergence due to the second order nature of the integrator. These tests have been run in all three directions with the waves moving in the positive or negative directions and gotten identical results.

\begin{figure}[ht!]
    % \epsscale{0.5}
    \plotone{assets/3-mhd-tests/linear_convergence.pdf}
    \caption{Linear Wave Convergence of all four MHD waves using PLM and PPM reconstruction. \input{|python ../python/get_links.py 'linear_wave_convergence'}}
    \label{fig:linear-wave-convergence}
\end{figure}

\subsection{Circularly Polarized Alfv\'en Wave}
\label{sec:cpaw}

The circularly polarized Alfv\'en wave is a non-linear wave that allows one to test the codes accuracy in the non-linear regime with the quantitative benefits of a regular wave test. The tests use a domain of $3.0\times1.5\times1.5$ and a resolution of $2N\times N \times N$ where $N$ goes from 8 to 256 in powers of 2 and periodic boundary conditions. The wave is set to travel at an oblique angle the grid making this a fully 3D test.

In coordinate system aligned with the movement of the wave the initial conditions are 
$\rho = 1.0$,
$P = 0.1$,
$v_x = (0,-1)$ for traveling or standing waves respectively,
$v_y = A \sin{\frac{2\pi x}{\lambda}}$,
$v_z = A \cos{\frac{2\pi x}{\lambda}}$,
$B_x = 1.0$,
$B_y = A \sin{\frac{2\pi x}{\lambda}}$,
$B_z = A \cos{\frac{2\pi x}{\lambda}}$,
where the amplitude of the wave $A = 0.1$ and the wavelength $\lambda = 1.0$. These coordinates are then rotated with the following rotation

\begin{eqnarray}
    x\prime = x \cos\alpha\cos\beta - y \sin\beta - z \sin\alpha\cos\beta \nonumber \\
    y\prime = x \cos\alpha\sin\beta + y \cos\beta - z \sin\alpha\sin\beta \nonumber \\
    z\prime = x \sin\alpha + z \cos\alpha \nonumber
\end{eqnarray}

with $\sin\alpha = 2/3$ and $\sin\beta = 1/\sqrt{5}$. This ensures the domain is fully periodic through the boundaries and the wave can travel (or stand) indefinitely. The magnetic fields are initialized with the vector potential to ensure initial divergence is zero to round off. The waves are then run for a single period and the L2 norm of the L2 error vector is plotted in Figure \ref{fig:cpaw} using the same method as in Section \ref{sec:lwc}. It's especially interesting to note that the improvement in accuracy that PPM brought over PLM in the linear waves is absent in this non-linear test.

These Alfv\'en waves are subject to a parametric instability \citep{del_zanna_parametric_2001} which should not be present for these initial conditions. However, the truncation error will result in small variations in the magnetic pressure which will drive low amplitude compression waves \citep{stone_athena_2008}. 

\begin{figure}[ht!]
    % \epsscale{0.5}
    \plotone{assets/3-mhd-tests/cpaw_convergence.pdf}
    \caption{Circularly Polarized Alfv\'en Wave Convergence using PLM and PPM reconstruction. \input{|python ../python/get_links.py 'cpaw'}}
    \label{fig:cpaw}
\end{figure}

\subsection{Advecting Field Loop}
\label{sec:afl}

In this test we advect a tilted spherical current loop across the domain at an oblique angle to the grid. This test requires particularly accurate balancing of the non-zero components of the induction equation. It also has zero magnetic field outside the spherical current loop, as the current loops moves across the grid those cells that are no longer in the loop should return to zero; within round off. It is also a good test of the dissipation of the magnetic field as the magnetic pressure should remain constant.

The initial conditions for this test are most easily described using the vector potential. The background state is
$\rho = 1.0$,
$P = 1.0$,
$v_x = 1.0$,
$v_y = 1.0$,
$v_z = 2.0$,
$B_x = 0$,
$B_y = 0$,
$B_z = 0$.

In the central region the state is given by the following vector potential that we have chosen such that $A_x = 0$,
\begin{equation}
    A_y = A_z = 
    \begin{cases}
        A \left( R - r \right),& \text{for}\; r < R\\
        0,              & \text{otherwise}
    \end{cases}
\end{equation}

where $r$ is the Euclidean distance from the center of the domain. Note that since the vector potential is along the vertices of the cells $A_y$ and $A_z$ will never have the same value at the same position since they are not stored at identical positions. The test is conducted in a grid of $N\times N\times 2N$ cells for $N=(32, 64, 128, 256)$ with a domain of $1.0\times1.0\times2.0$ centered at zero and evolved for two periods; $t_{max} = 2.0$.

In Figure \ref{fig:afl} the mean of cell centered $B^2$ normalized to the initial value is plotted to show the convergence of the dissipation rate. The dissipation rate is comparable to those found in the literature \citep{stone_athena_2008} and improves at approximately first order. Figure \ref{fig:afl} also shows the maximum divergence in the domain as a function of time. Throughout the entire evolution it remains near round off and, after an initial rise, remains fairly constant. The zero magnetic field region outside of the current loop also remains near zero throughout the entire evolution of the problem.
 
\begin{figure}[ht!]
    % \epsscale{0.5}
    \plotone{assets/3-mhd-tests/afl.pdf}
    \caption{Evolution of tilted spherical magnetic field loop through two full periods. Mean of $B^2$ normalized to the initial value as a function of time (left) and the maximum divergence in the domain as a function of time (right). \input{|python ../python/get_links.py 'afl'}}
    \label{fig:afl}
\end{figure}

\subsection{MHD Riemann Problems}
\label{sec:riemann}

MHD Riemann problems are staple tests for new MHD codes and methods due to their mix of different flow types and extreme conditions. While there are a variety of different MHD Riemann problems in the literature \citep{brio_wu_1988, einfeldt_1991, ryu_jones_1995, dai_woodward_1998} we have chosen five that present a variety of challenging cases. 

The Riemann problem is defined as having a single state for half of the domain which suddenly switches to a different state on the other half of the domain. All of Riemann problems here have a domain of $1\times1\times1$ and resolution of $512\times16\times16$ and are run until the $t_{max}$ for that particular Riemann problem. All have been run in all three spatial directions with both possible orientations of the two states and achieved identical results. The details of each left and right state are in Table \ref{table:riemann} and the details of each Riemann problem are discussed in the captions of Figures \ref{fig:brio-and-wu}-\ref{fig:einfeldt}.

% Can add a * after "deluxetable" in begin and end to let it span 2 columns

%% The values (usually only l,r and c) in the last part of
%% \begin{deluxetable}{} command tell LaTeX how many columns
%% there are and how to align them.
\begin{deluxetable*}{lccccccccccccccccc}
    \label{table:riemann}
    
    %% Over-ride the default font size
    %% Use Default (12pt)
    
    %% Use \tablewidth{?pt} to over-ride the default table width.
    %% If you are unhappy with the default look at the end of the
    %% *.log file to see what the default was set at before adjusting
    %% this value.
    
    %% This is the title of the table.
    \tablecaption{Riemann Problem Initial Conditions}
    
    %% The \tablehead gives provides the column headers.  It
    %% is currently set up so that the column labels are on the
    %% top line and the units surrounded by ()s are in the 
    %% bottom line.  You may add more header information by writing
    %% another line between these lines. For each column that requries
    %% extra information be sure to include a \colhead{text} command
    %% and remember to end any extra lines with \\ and include the 
    %% correct number of &s.
    \tablehead{\colhead{Riemann Problem} & \colhead{$\gamma$} & \colhead{$t_{max}$} & \colhead{$B_x$} & 
    \colhead{$\rho_L$} & \colhead{$P_L$} & \colhead{$v_{x,L}$} & \colhead{$v_{y,L}$} & \colhead{$v_{z,L}$} & \colhead{$B_{y,L}$} & \colhead{$B_{z,L}$} & 
    \colhead{$\rho_R$} & \colhead{$P_R$} & \colhead{$v_{x,R}$} & \colhead{$v_{y,R}$} & \colhead{$v_{z,R}$} & \colhead{$B_{y,R}$} & \colhead{$B_{z,R}$}}
    
    %% All data must appear between the \startdata and \enddata commands
    \startdata
    % Riemann Problem      gamma           t_max  B_x                       rho_L  P_L    V_x,L  V_y,L  V_z,L B_y,L                       B_z,L                     rho_R   P_R    V_x,R   V_y,R V_z,R B_y,R                     B_z,R
    Brio \& Wu           & 2             & 0.1  & 0.75                    & 1    & 1    & 0    & 0    & 0   & 1                         & 0                       & 0.128 & 0.1  & 0     & 0   & 0   & -1                      & 0                       \\
    Dai \& Woodward      & $\frac{5}{3}$ & 0.2  & $\frac{2}{\sqrt{4\pi}}$ & 1.08 & 0.95 & 1.2  & 0.01 & 0.5 & $\frac{3.6}{\sqrt{4\pi}}$ & $\frac{2}{\sqrt{4\pi}}$ & 1     & 1    & 0     & 0   & 0   & $\frac{4}{\sqrt{4\pi}}$ & $\frac{2}{\sqrt{4\pi}}$ \\
    Ryu \& Jones 1a      & $\frac{5}{3}$ & 0.08 & $\frac{5}{\sqrt{4\pi}}$ & 1    & 20   & 10   & 0    & 0   & $\frac{5}{\sqrt{4\pi}}$   & 0                       & 1     & 1    & -10   & 0   & 0   & $\frac{5}{\sqrt{4\pi}}$ & 0                       \\
    Ryu \& Jones 4d      & $\frac{5}{3}$ & 0.16 & 0.7                     & 1    & 1    & 0    & 0    & 0   & 0                         & 0                       & 0.3   & 0.2  & 0     & 0   & 1   & 1                       & 0                       \\
    Einfeldt Rarefaction & 1.4           & 0.16 & 0                       & 1    & 0.45 & -2   & 0    & 0   & 0.5                       & 0                       & 1     & 0.45 & 2     & 0   & 0   & 0.5                     & 0                       \\
    \enddata
    
    %% Include any \tablenotetext{key}{text}, \tablerefs{ref list},
    %% or \tablecomments{text} between the \enddata and 
    %% \end{deluxetable} commands
    
    %% General table comment marker
    \tablecomments{The $L/R$ subscripts indicate that it is the left/right state. $B_x$ is always the same in both states.}
    
\end{deluxetable*}

\begin{figure}[ht!]
    % \epsscale{0.5}
    \plotone{assets/3-mhd-tests/b&w.pdf}
    \caption{Brio \& Wu Shock Tube solution. This is essentially the Sod shock tube with a magnetic field. However, this shock tube is an excellent stress test for the PPM reconstruction as PPM methods tend to create large oscillations in the solution due to the slowly moving shocks. To minimize these oscillations we implemented a new PPM reconstruction algorithm based on \cite{felker_2020} which reduced the oscillations to a reasonable level. No oscillation is present when using PLM reconstruction.
    \input{|python ../python/get_links.py 'b&w'}}
    \label{fig:brio-and-wu}
\end{figure}

\begin{figure}[ht!]
    % \epsscale{0.5}
    \plotone{assets/3-mhd-tests/d&w.pdf}
    \caption{Dai \& Woodward Shock Tube (also called Ryu \& Jones 2a) solution. This shock tube produces all seven possible MHD waves: from left to right a fast shock, Alfvén wave, slow shock, contact discontinuity, slow shock, Alfvén wave, and fast shock. This makes it an excellent laboratory for checking that the full spread of mave modes are well resolved
    \input{|python ../python/get_links.py 'd&w'}}
    \label{fig:dai-and-woodward}
\end{figure}

\begin{figure}[ht!]
    % \epsscale{0.5}
    \plotone{assets/3-mhd-tests/rj1a.pdf}
    \caption{Ryu \& Jones 1a Shock Tube solution. This shock tube has a fairly simple wave structure without any spikes which makes it easy to use for debugging purposes and a good test for over/undershoot of the solution near discontinuities.
    \input{|python ../python/get_links.py 'rj1a'}}
    \label{fig:rj-1a}
\end{figure}

\begin{figure}[ht!]
    % \epsscale{0.5}
    \plotone{assets/3-mhd-tests/rj4d.pdf}
    \caption{Ryu \& Jones 4d Shock Tube solution. This test features a switch-on slow shock. Switch-on waves increase the strength of the transverse magnetic field while reducing the thermal pressure to maintain energy conservation. This is a simplified example of a potential type of magnetic field amplification in galaxies and as such it is important that we replicate it accurately. A "switch-off" wave does the inverse.
    \input{|python ../python/get_links.py 'rj4d'}}
    \label{fig:rj-4d}
\end{figure}

\begin{figure}[ht!]
    % \epsscale{0.5}
    \plotone{assets/3-mhd-tests/einfeldt.pdf}
    \caption{MHD Einfeldt Strong Rarefaction solution. This Riemann problem simulates a strong outflow and central vacuum state. These diverging fluids lead to an extremely strong and fast rarefaction where the energy is dominated by kinetic energy and as such can often reveal issues in code accuracy since it can lead to nonphysical states with negative density or negative internal energy. High values of the outflow velocity ($V_{out}\ge3$) can lead to spurious oscillations so $V_{out} = 2$ was chosen\citep{charm_2011}.
    \input{|python ../python/get_links.py 'einfeldt'}}
    \label{fig:einfeldt}
\end{figure}

\subsection{MHD Blast Wave in a Strongly Magnetized Medium}
\label{sec:mhd-blast}

Blast waves in different forms are staple tests for hydrodynamics and MHD codes. They combine strong shocked flows, smooth flows, and, in MHD, strong magnetic fields. The results are qualitative rather than quantitative but thoroughly test the robustness of the algorithm and act act as an excellent regression test for our automated testing (see Section \ref{sec:testing}). In this test $\beta = 0.2$, like \cite{stone_2009} we note instability if $\beta$ is decreased by a factor of 10. This issue could possibly be resolved by integrating the internal energy separate from the total energy using the dual energy formalism in Cholla.

The background state is
$\rho = 1.0$,
$P = 0.1$,
$v_x = 0.0$,
$v_y = 0.0$,
$v_z = 0.0$,
$B_x = 1/\sqrt{2}$,
$B_y = 1/\sqrt{2}$,
$B_z = 0.0$,
and the over pressure region is a central sphere of size $R = 0.1$ which has $P=10.0$.

The test was then run with a domain of $1\times1.5\times1$ and resolution of $200\times300\times200$ until $t = 0.2$. Figure \ref{fig:blast} shows contours in the density and magnetic energy in an $x-y$ slice through the center of the domain. The contours are smooth and symmetric and show clear elongation of the blast wave rarefaction parallel to the magnetic field. The blast wave propagates slowly parallel to the magnetic field but much more rapidly perpendicular to the magnetic field

\begin{figure}[ht!]
    % \epsscale{0.5}
    \plotone{assets/3-mhd-tests/mhd-blast.pdf}
    \caption{Contour plot of the MHD blast wave test at $t=0.2$. 30 evenly spaced contours are shown in an $x-y$ slice through the center of the domain. \input{|python ../python/get_links.py 'mhd-blast'}}
    \label{fig:blast}
\end{figure}

\subsection{Orszag-Tang Vortex}
\label{sec:otv}

The Orszag-Tang vortex is a standard 2D MHD test from (CITE Orszag \& Tang (1979)). While it does not provide a quantitative measure of accuracy like the linear wave tests or a test of the robustness of the method like the MHD blast wave it does have a very complex flow that is sensitive to changes in the integrator; making it ideal for regression testing.

The test was conducted with a periodic domain of $1\times1\times1$ and resolution of $192\times192\times192$ until $t = 0.5$ with the following initial conditions: 
$\rho = 25 / \left( 36 \pi \right)$,
$P    =  5 / \left( 12 \pi \right)$,
$v_x  = \sin 2\pi y$,
$v_y  = -\sin 2\pi x$,
$v_z  = 0.0$,
$A_x  = 0.0$,
$A_y  = 0.0$,
$A_z  = \left( B_0/4\pi \right) \left( \cos{4\pi x} + 2 \cos{2\pi y} \right)$, with $B_0 = 1/sqrt{4\pi}$.

The results, plotted in Figure \ref{fig:otv}, can be compared directly to Figure 22 in \cite{stone_athena_2008} as a qualitative check for correctness of the flow structure.

\begin{figure}[ht!]
    % \epsscale{0.5}
    \plotone{assets/3-mhd-tests/orszag-tang-vortex.pdf}
    \caption{Contour plot of the Orszag-Tang Vortex at $t=0.5$. Thirty evenly spaced contours are shown for each plot in an $x-y$ slice through the center of the domain.  \input{|python ../python/get_links.py 'otv'}}
    \label{fig:otv}
\end{figure}

\section{MHD Performance Tests}
\label{sec:mhd-perf-tests}

Given that Cholla is intended to be a massively parallel code the scaling properties warrant discussion. However, we will focus on weak scaling rather than strong scaling as our primary goal is to be able to simulate as many cells as possible and because strong scaling rapidly reduces the number of cells per GPU to the point where the whole GPU cannot be utilized. The results of the weak scaling tests are shown in Figures \ref{fig:scaling-cells-per-second} and \ref{fig:scaling-ms-per-gpu}. 

Our primary scaling tests were performed on the Frontier Supercomputer at the Oak Ridge Leadership Computing Facility. Frontier utilizes AMD Radeon Instinct MI250X GPUs, a notable feature of these GPUs is that they contain two Graphics Compute Dies (GCDs) each of which is essentially a separate GPU and treated as such in software. As such, for the sake of clear comparison to other systems, we will refer to each GCD as a single GPU for the remainder of this paper. On Frontier Cholla updates $1.7\times 10^8$ cells per second per GPU up to 32,768 GPUs and $1.5\times 10^8$ cells per second per GPU above 32,768 GPUs up to 74,088 GPUs. The lower performance per GPU occasionally appears on smaller tests and tests with the lowered performance also demonstrate much larger variance in performance per GPU. As such our suspicion is that the lower performance is caused by a small number of slower GPUs and is not a feature of Cholla.

With a single rank there is no MPI overhead. As the number of ranks grows the MPI overhead quickly stabilizes at around 15ms with a moderate increase when running on the full size of Frontier. Perhaps most importantly, the VL+CT integrator scales almost perfectly and takes up most of the time on each time steps, dominating over the MPI communication time by nearly a factor of 10. Overall this shows that Cholla has excellent weak scaling up to the full size of Frontier.

All the weak scaling tests here were performed with a slow magnetosonic wave perturbation (described in Section \ref{sec:lwc}), periodic boundary conditions, in double precision, and with $256^3$ cells per MPI rank; each rank corresponds to one GPU. This wave was then evolved for 103 time steps and the results averaged, we exclude setup and tear down time as they are disproportionately significant for short runs like this. This wave was chosen as it is generally the most challenging of the waves to reproduce accurately in our experience. This problem requires that every cell be updated and uses the third order piecewise parabolic reconstruction method in the characteristic variables. Overall this means that this test is a reasonably worst case scenario.


% Performance on different hardware: CRC V100+PPC, CRC A100+x86, Summit, Frontier, Grace Hopper?

\begin{figure}[ht!]
    % \epsscale{0.5}
    \plotone{assets/3-mhd-tests/scaling_tests_ms_per_gpu.pdf}
    \caption{Weak scaling performance of Cholla MHD with the contribution of the VL+CT integrator and MPI communication highlighted.  \input{|python ../python/get_links.py 'scaling_plot'}}
    \label{fig:scaling-ms-per-gpu}
\end{figure}

\begin{figure}[ht!]
    % \epsscale{0.5}
    \plotone{assets/3-mhd-tests/scaling_tests_cells_per_second.pdf}
    \caption{Weak scaling performance of Cholla MHD. Cell updates per second per MPI rank/GPU is shown.  \input{|python ../python/get_links.py 'scaling_plot'}}
    \label{fig:scaling-cells-per-second}
\end{figure}

