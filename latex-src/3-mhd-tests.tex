\section{MHD Tests}
\label{sec:mhd-tests}

\subsection{Linear Wave Convergence}
\label{sec:lwc}

The propagation of the four MHD linear waves provide an excellent quantitative measure of the accuracy of computation methods \citep{stone_2009}. The tests use a domain of $1.0\times1.0\times1.0$ and a resolution of $N\times16\times16$ where $N$ goes from 16 to 512 in powers of 2. The equation of the wave is

\begin{equation}
    q = \overline{q} + A R_w \sin{\frac{2\pi x}{\lambda}}
\end{equation}

where $q$ is the conserved variable, $\overline{q}$ is the mean background state, $A=10^{-6}$ is the amplitude of the wave, $R_w$ is the right eigenvector in conserved variables for the wave mode $w$, $x$ is the position, and $\lambda=1$ is the wavelength of the wave. The adiabatic index $\gamma$ is $5/3$ and the background state is: 
$\overline{\rho}=1.0$,
$\overline{v_x}=\overline{v_y}=\overline{v_z}=0$ (except for the contact wave where $\overline{v_x} = 1$),
$\overline{P}=1/\gamma$,
$\overline{B_x}=1$,
$\overline{B_y}=1.5$,
$\overline{B_z}=0$ 
and the right eigenvectors for this state are given in Appendix A of \cite{gardiner_unsplit_2008}. 

The wave is propagated for one period and then the error between the initial and final state is the L2 norm of the L1 error vector. First we compute the L1 norm of the absolute difference for each conserved variable between the initial and final state

\begin{equation}
    \delta q_s = \frac{1}{n_x n_y n_z} \sum_{i,j,k} \mid q^f_{i,j,k,s} - q^i_{i,j,k,s} \mid
\end{equation}

where $q_s$ is a specific conserved variable. We then compute the L2 norm of this vector of L1 norms

\begin{equation}
    \mid \mid \delta q \mid \mid = \sqrt{\sum_s \left( \delta q_s \right)^2}.
\end{equation}

These L2 errors are plotted in \autoref{fig:linear-wave-convergence} for both the PLM and PPM reconstructions. The results are comparable to the results in \cite{stone_2009} and demonstrate the expected second order convergence for this second order method. Using PPM does improve the accuracy of the method but maintains the second order convergence due to the second order nature of the integrator. These tests have been run in all three directions with the waves moving in the positive or negative directions and gotten identical results.

\begin{figure}[ht!]
    % \epsscale{0.5}
    \plotone{assets/3-mhd-tests/linear_convergence.pdf}
    \caption{Linear Wave Convergence of all four MHD waves using PLM and PPM reconstruction. 
    \img{../assets/github.png}
    \href{https://github.com/bcaddy/caddy-et-al-2023/blob/5fcdc30d0fdd93bd7cbf716c7fbc7142daedec20/python/linear-wave-convergence.py}{\img{../assets/github.png}}
    \input{|python ../python/get_links.py 'linear_wave_convergence'}}
    \label{fig:linear-wave-convergence}
\end{figure}

\subsection{Circularly Polarized Alfv\'en Wave}
\label{sec:cpaw}

\begin{figure}[ht!]
    % \epsscale{0.5}
    \plotone{assets/3-mhd-tests/cpaw_convergence.pdf}
    \caption{Circularly Polarized Alfven Wave Convergence plot. \input{|python ../python/get_links.py 'cpaw'}}
    \label{fig:cpaw}
\end{figure}

\subsection{Advecting Field Loop}
\label{sec:afl}

\subsection{MHD Riemann Problems}
\label{sec:riemann}

MHD Riemann problems are staple tests for new MHD codes and methods due to their mix of different flow types and extreme conditions. While there are a variety of different MHD Riemann problems in the literature \citep{brio_wu_1988, einfeldt_1991, ryu_jones_1995, dai_woodward_1998} we have chosen five that present a variety of challenging cases. 

The Riemann problem is defined as having a single state for half of the domain which suddenly switches to a different state on the other half of the domain. All of Riemann problems here have a domain of $1\times1\times1$ and resolution of $512\times16\times16$ and are run until the $t_{max}$ for that particular Riemann problem. All have been run in all three spatial directions with both possible orientations of the two states and achieved identical results. The details of each left and right state are in Table \ref{table:riemann} and the details of each Riemann problem are discussed in the captions of Figures \ref{fig:brio-and-wu}-\ref{fig:einfeldt}.

% Can add a * after "deluxetable" in begin and end to let it span 2 columns

%% The values (usually only l,r and c) in the last part of
%% \begin{deluxetable}{} command tell LaTeX how many columns
%% there are and how to align them.
\begin{deluxetable*}{lccccccccccccccccc}
    \label{table:riemann}
    
    %% Over-ride the default font size
    %% Use Default (12pt)
    
    %% Use \tablewidth{?pt} to over-ride the default table width.
    %% If you are unhappy with the default look at the end of the
    %% *.log file to see what the default was set at before adjusting
    %% this value.
    
    %% This is the title of the table.
    \tablecaption{Riemann Problem Initial Conditions}
    
    %% The \tablehead gives provides the column headers.  It
    %% is currently set up so that the column labels are on the
    %% top line and the units surrounded by ()s are in the 
    %% bottom line.  You may add more header information by writing
    %% another line between these lines. For each column that requries
    %% extra information be sure to include a \colhead{text} command
    %% and remember to end any extra lines with \\ and include the 
    %% correct number of &s.
    \tablehead{\colhead{Riemann Problem} & \colhead{$\gamma$} & \colhead{$t_{max}$} & \colhead{$B_x$} & 
    \colhead{$\rho_L$} & \colhead{$P_L$} & \colhead{$v_{x,L}$} & \colhead{$v_{y,L}$} & \colhead{$v_{z,L}$} & \colhead{$B_{y,L}$} & \colhead{$B_{z,L}$} & 
    \colhead{$\rho_R$} & \colhead{$P_R$} & \colhead{$v_{x,R}$} & \colhead{$v_{y,R}$} & \colhead{$v_{z,R}$} & \colhead{$B_{y,R}$} & \colhead{$B_{z,R}$}}
    
    %% All data must appear between the \startdata and \enddata commands
    \startdata
    % Riemann Problem      gamma           t_max  B_x                       rho_L  P_L    V_x,L  V_y,L  V_z,L B_y,L                       B_z,L                     rho_R   P_R    V_x,R   V_y,R V_z,R B_y,R                     B_z,R
    Brio \& Wu           & 2             & 0.1  & 0.75                    & 1    & 1    & 0    & 0    & 0   & 1                         & 0                       & 0.128 & 0.1  & 0     & 0   & 0   & -1                      & 0                       \\
    Dai \& Woodward      & $\frac{5}{3}$ & 0.2  & $\frac{2}{\sqrt{4\pi}}$ & 1.08 & 0.95 & 1.2  & 0.01 & 0.5 & $\frac{3.6}{\sqrt{4\pi}}$ & $\frac{2}{\sqrt{4\pi}}$ & 1     & 1    & 0     & 0   & 0   & $\frac{4}{\sqrt{4\pi}}$ & $\frac{2}{\sqrt{4\pi}}$ \\
    Ryu \& Jones 1a      & $\frac{5}{3}$ & 0.08 & $\frac{5}{\sqrt{4\pi}}$ & 1    & 20   & 10   & 0    & 0   & $\frac{5}{\sqrt{4\pi}}$   & 0                       & 1     & 1    & -10   & 0   & 0   & $\frac{5}{\sqrt{4\pi}}$ & 0                       \\
    Ryu \& Jones 4d      & $\frac{5}{3}$ & 0.16 & 0.7                     & 1    & 1    & 0    & 0    & 0   & 0                         & 0                       & 0.3   & 0.2  & 0     & 0   & 1   & 1                       & 0                       \\
    Einfeldt Rarefaction & 1.4           & 0.16 & 0                       & 1    & 0.45 & -2   & 0    & 0   & 0.5                       & 0                       & 1     & 0.45 & 2     & 0   & 0   & 0.5                     & 0                       \\
    \enddata
    
    %% Include any \tablenotetext{key}{text}, \tablerefs{ref list},
    %% or \tablecomments{text} between the \enddata and 
    %% \end{deluxetable} commands
    
    %% General table comment marker
    \tablecomments{The $L/R$ subscripts indicate that it is the left/right state. $B_x$ is always the same in both states.}
    
\end{deluxetable*}

\begin{figure}[ht!]
    % \epsscale{0.5}
    \plotone{assets/3-mhd-tests/b&w.pdf}
    \caption{Brio \& Wu Shock Tube solution. This is essentially the Sod shock tube with a magnetic field. However, this shock tube is an excellent stress test for the PPM reconstruction as PPM methods tend to create large oscillations in the solution due to the slowly moving shocks. To minimize these oscillations we implemented a new PPM reconstruction algorithm based on \cite{felker_2020} which reduced the oscillations to a reasonable level. No oscillation is present when using PLM reconstruction.
    \input{|python ../python/get_links.py 'b&w'}}
    \label{fig:brio-and-wu}
\end{figure}

\begin{figure}[ht!]
    % \epsscale{0.5}
    \plotone{assets/3-mhd-tests/d&w.pdf}
    \caption{Dai \& Woodward Shock Tube (also called Ryu \& Jones 2a) solution. This shock tube produces all seven possible MHD waves: from left to right a fast shock, Alfvén wave, slow shock, contact discontinuity, slow shock, Alfvén wave, and fast shock. This makes it an excellent laboratory for checking that the full spread of mave modes are well resolved
    \input{|python ../python/get_links.py 'd&w'}}
    \label{fig:dai-and-woodward}
\end{figure}

\begin{figure}[ht!]
    % \epsscale{0.5}
    \plotone{assets/3-mhd-tests/rj1a.pdf}
    \caption{Ryu \& Jones 1a Shock Tube solution. This shock tube has a fairly simple wave structure without any spikes which makes it easy to use for debugging purposes and a good test for over/undershoot of the solution near discontinuities.
    \input{|python ../python/get_links.py 'rj1a'}}
    \label{fig:rj-1a}
\end{figure}

\begin{figure}[ht!]
    % \epsscale{0.5}
    \plotone{assets/3-mhd-tests/rj4d.pdf}
    \caption{Ryu \& Jones 4d Shock Tube solution. This test features a switch-on slow shock. Switch-on waves increase the strength of the transverse magnetic field while reducing the thermal pressure to maintain energy conservation. This is a simplified example of a potential type of magnetic field amplification in galaxies and as such it is important that we replicate it accurately. A "switch-off" wave does the inverse.
    \input{|python ../python/get_links.py 'rj4d'}}
    \label{fig:rj-4d}
\end{figure}

\begin{figure}[ht!]
    % \epsscale{0.5}
    \plotone{assets/3-mhd-tests/einfeldt.pdf}
    \caption{MHD Einfeldt Strong Rarefaction solution. This Riemann problem simulates a strong outflow and central vacuum state. These diverging fluids lead to an extremely strong and fast rarefaction where the energy is dominated by kinetic energy and as such can often reveal issues in code accuracy since it can lead to nonphysical states with negative density or negative internal energy. High values of the outflow velocity ($V_{out}\ge3$) can lead to spurious oscillations so $V_{out} = 2$ was chosen\citep{charm_2011}.
    \input{|python ../python/get_links.py 'einfeldt'}}
    \label{fig:einfeldt}
\end{figure}

\subsection{MHD Blast Wave in a Strongly Magnetized Medium}
\label{sec:mhd-blast}

\begin{figure}[ht!]
    % \epsscale{0.5}
    \plotone{assets/3-mhd-tests/mhd-blast.pdf}
    \caption{Contour plot of the MHD blast wave test \input{|python ../python/get_links.py 'mhd-blast'}}
    \label{fig:blast}
\end{figure}

\subsection{Orszag-Tang Vortex}
\label{sec:otv}

\begin{figure}[ht!]
    % \epsscale{0.5}
    \plotone{assets/3-mhd-tests/orszag-tang-vortex.pdf}
    \caption{Contour plot of the Orszag-Tang Vortex \input{|python ../python/get_links.py 'otv'}}
    \label{fig:otv}
\end{figure}

\section{MHD Performance Tests}
\label{sec:mhd-perf-tests}

% Performance on different hardware: CRC V100+PPC, CRC A100+x86, Summit, Frontier, Grace Hopper?

\subsection{Weak Scaling}

\subsection{Cell Updates Per Second Per GPU}

