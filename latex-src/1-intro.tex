\section{Introduction}
\label{sec:intro}

% ==============================================================================
% Outline:
% - why do big sims matter?
%   - esp fast efficient MHD codes
%   - what kinds of sims?
%     - giant turbulent boxes, mhd outflows of galaxies
% - why this code? 
%   - why a new code? Cholla is fast, scales well, GPUs are fast, all big new 
%     computers are gpu based
%   - finite volume + CT, what other methods? Why is divergence cleaning bad
%   - cite lots of other codes and why is our code different
%   - why do we use this framework, what other options are there
%   - we have a testing framework that allows for scalable development, makes 
%     it easier and faster to dev since they don't have to worry about breaking 
%     other peoples stuff
% - Testing/CI stuff
% - outline paper
%   - alude to performance and testing at scale
% ==============================================================================

% - why do big sims matter?
%   - esp fast efficient MHD codes
Over the past decade is has become increasingly clear that magnetohydrodynamics (MHD) plays a significant role in astrophysical phenomenon. This role is both direct, e.g. MHD waves in stellar corona, and indirect, e.g. cosmic ray propagation. Modern MHD methods are fast, sophisticated, and robust but nonetheless MHD simulations remain very computationally expensive. This computational expense is in large part due to the fact that MHD turbulent dynamos operate across all length scales and so high resolution simulations are critically important for accurately simulating astrophysical MHD. The computational expense of MHD simulations along with the advent of Graphics Processing Units (GPUs) as the primary source of computational power in new cutting edge supercomputers\footnote{https://www.top500.org} necessitates the development of a GPU based astrophysical MHD simulation code. 

% - why this code? 
%   - why a new code? Cholla is fast, scales well, GPUs are fast, all big new computers are gpu based
%   - finite volume + CT, what other methods? Why is divergence cleaning bad
%   - cite lots of other codes and why is our code different
%   - why do we use this framework, what other options are there
%   - we have a testing framework that allows for scalable development, makes it easier and faster to dev since they don't have to worry about breaking other peoples stuff
The Cholla code (Computational Hydrodynamics On paraLLel Architectures)\citep{schneider_2015} is a massively parallel, finite volume, fixed grid, GPU native, astrophysical hydrodynamics code that was designed from the ground up to be extended to MHD and run on massive GPU supercomputers. This work presents the MHD extension of Cholla. The MHD integrator is based off the Van Leer + Constrained Transport (VL+CT) integrator of \cite{stone_2009} with modifications for GPUs. It also utilizes the HLLD Riemann approximate solver and includes second and third order reconstruction in the characteristic variables.

%   - what kinds of sims?
%     - giant turbulent boxes, mhd outflows of galaxies
Cholla with MHD allows the simulation of previously unreachable domains. The VL+CT integrator provides high accuracy results with divergences that are zero to round off error. It is performant enough that a single GPU can run a $256^3$ cell MHD simulation rapidly or many GPUs can work together on a cluster for simulations of up to $\approx 10,000^3$ cells on Frontier\footnote{https://www.olcf.ornl.gov/frontier/}. This allows MHD simulations of entire galaxies with a constant resolution of a few parsecs per cell, turbulent box simulations with resolutions of XYZ, or parameter studies of lower resolution to be computed rapidly.

% - Testing/CI stuff
We also present on the usage of an automated testing/continuous integration (CI) pipeline in Cholla. As the complexity of Cholla grew and more developers and collaborators started working with Cholla the necessity of having robust tests that are easy to run became paramount. Our approach works using a hybrid of several different tools and was designed from the ground up to be scalable from a single GPU all the way up to an exascale machine. This has allowed much more rapid and confident development on Cholla with less risk of bugs.

% - outline paper
%   - alude to performance and testing at scale
In Section \ref{sec:methods} we describe the VL+CT algorithm in detail along with the modifications required to utilize GPUs. 
In Section \ref{sec:mhd-tests} we show a suite of MHD test problems to demonstrate the correctness and accuracy of the code and demonstrate the performance and weak scaling on up to 74,088 GPUs.
In Section \ref{sec:testing} we discuss the new testing framework. 
We conclude in Section \ref{sec:summary}. 

In figure captions the GitHub icon, \img{../assets/github.png}, serves as a link to the version of the python script which generated that figure. These scripts, and the associated GitHub repository, have sufficient information to reproduce the figures shown in this paper.